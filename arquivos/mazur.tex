% ==== LAYOUT PÁGINA ====
\geometry{
    paper=a4paper,
    vmargin=3cm,
    hmargin=2cm,
    twocolumn=true
}

% ==== CORES ====
\definecolor{mazurblue}{HTML}{0067b1}
\definecolor{mazurdarkblue}{HTML}{00478d}
\definecolor{mazurgray}{HTML}{b0afb5}
\definecolor{mazurlightblue}{HTML}{f6fbfe}
\definecolor{mazurlightgreen}{HTML}{F6FAF3}

% ==== FONTES ====
\setmainfont{Minion Pro}
\setmathfont{XITS Math}
\setsansfont{HelveticaNeueMedium.ttf}[
    Path = C:/Users/jfcmp/Downloads/HelveticaNeue/,
    BoldFont = HelveticaNeueBold.ttf
]
% Fonte usada no texto "SOLUTION" dentro das caixas de "Exercise"
\newfontfamily\solutionfont{HelveticaNeueSolution.otf}[
    Path = C:/Users/jfcmp/Coisas do Jota/Vault-1/Programas e Tutoriais/Configurações e Predefinições/Mais fontes/HelveticaNeue/
]

% ==== SÍMBOLOS ESPECIAIS ====
% Define um comando "\checkmarkmazur" que coloca em, linha, um gráfico vetorizado de um checkmark.
\newcommand{\checkmarkmazur}{\includesvg[height=.65\baselineskip,width=\baselineskip,keepaspectratio]{check.svg}}

% ==== AMBIENTES ====

% Define o ambiente de um exercício do mazur
% Contador de exercícios (definido por seção)
\newcounter{mazurex}[section]

% Construção de um novo ambiente
\newenvironment{mazurex}[1][]  % Nome, No Argumentos, Valor default do argumento #1
{ % O que vem antes do conteúdo do ambiente.
    \refstepcounter{mazurex} % incrementa o contador
    \par\medskip\noindent % quebra a linha e dá um espaço vertical
    {\bfseries\sffamily\color{mazurdarkblue} Exercise \thesection.\themazurex~#1} % escreve o rótulo (com numeração) e o argumento opcional (nome do exercício)
    \rmfamily\par\medskip\noindent % quebra linha e dá um espaço
    \begin{tcolorbox}[enhanced, 
        frame hidden, % Desabilita as bordas
        arc = 0mm, % Ângulos retos
        top=1mm, bottom=1mm, right=1mm, left=1mm, % Margens
        colback = mazurlightblue, % Cor do fundo
        borderline horizontal={1.5pt}{1pt}{mazurgray}] % Adiciona retas
    \small % texto dentro da caixa tem tamanho \small
}
{ % O que vem depois do conteúdo do ambiente.
    \end{tcolorbox}\medskip
}

% Ambiente onde a questão será colocada (para usar dentro de mazurex).
% É somente o texto, sem nada.
\newenvironment{mazurq}{}{}

% Ambiente de solução (para usar dentro de mazurex e depois de mazurq)
\newenvironment{mazursol}
{ % O que vem antes do texto da solução
    \medskip
    {\solutionfont % Fonte customizada
    \addfontfeatures{LetterSpace=7.0} % Espaçamento da fonte
    \color{mazurdarkblue}\scriptsize SOLUTION}} % Cor e Tamanho
{}

% Exemplo
%   \begin{mazurex}[Throwing Dice]
%       \begin{mazurq}
%       ...
%       \end{mazurq}
%       
%       \begin{mazursol}
%       ... \checkmarkmazur
%       \end{mazursol}
%   \end{mazurex}

% tcolorbox p questões com mãozinha
\newtcolorbox[auto counter,number within=section]% Contador dentro de seção
{questionbox}[1][]% Nome: questionbox e um argumento opcional.
{
enhanced,
colback=mazurlightgreen, % Cor do Fundo
frame empty, % Não tem moldura
fonttitle=\qquad\sffamily\bfseries, % Fonte do título
coltitle=black,
title=\thetcbcounter, % Título é somente o contador da caixa
detach title, % O título fica de fora da moldura
before upper=\tcbtitle\ , % Antes do texto da parte superior
fontupper=\small\ , % Fonte da parte superior
#1, % Rótulo opcional
arc=0mm, % Ângulos retos
left=1mm, right=1mm, top=1mm, bottom=1mm, % Margens
overlay={\node[anchor=south west,outer sep=0pt, yshift=-17pt,xshift=7pt] at (frame.north west) {\includesvg[height=6.5mm]{mazur_hand.svg}}; }} % Coloca a mãozinha


% ==== HYPERLINKS ====
\hypersetup{
    colorlinks = true,
    urlcolor = cyan,
    linkcolor = mazurblue
}

% ==== TÍTULOS ====
\titleformat{\section}[hang]{\color{mazurdarkblue}\sffamily\bfseries\LARGE}{\huge\thetitle}{.5cm}{}

\titleformat{\subsection}[hang]{\color{mazurdarkblue}\sffamily\bfseries\Large}{\Large\thesubsection}{.1cm}{}

% ==== REFERÊNCIAS CRUZADAS ====
\crefformat{figure}{\footnotesize\sffamily\bfseries#2Figure \thesection.#1#3}
\Crefformat{figure}{\small\sffamily\bfseries#2Figure \thesection.#1#3}

% ==== FLOATS ====
\floatplacement{figure}{H}

% ==== PATH FIGURAS ====
\graphicspath{{./imgs/}}

% ==== LEGENDAS ====
\DeclareCaptionFormat{myformat}{#1#2#3\par}
\DeclareCaptionLabelFormat{mylabformat}{\color{mazurblue}\footnotesize\sffamily\bfseries #1 \thesection.#2}
\DeclareCaptionTextFormat{mytextformat}{\footnotesize#1}
\captionsetup{format=myformat,
labelformat=mylabformat,
textformat=mytextformat,
labelsep=space,
singlelinecheck = false,
justification=raggedright}

% ==== RODAPÉ ====
\DefineFNsymbols*{myfnsymbols}{{\rotatebox[origin=c]{-30}{$\star$}}\ast}
\setfnsymbol{myfnsymbols}